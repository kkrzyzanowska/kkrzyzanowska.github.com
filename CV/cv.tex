
\documentclass[a4paper,12pt]{article}

\usepackage{ucs}
\usepackage[T1]{polski}
\usepackage[utf8x]{inputenc}
\usepackage{geometry}
\usepackage{graphicx}
%\usepackage{beramono}

\pagestyle{empty}

\newenvironment{sekcja}
{\begin{tabular}{r|l}}
{\end{tabular}}

%\usepackage{hyperref}
\usepackage{color}
\usepackage[usenames,dvipsnames]{xcolor}
\usepackage[final, %override "draft which means "do nothing"
colorlinks, %rather than outlining them in boxes
linkcolor=Sepia, %override truly awful colour choices
citecolor=Sepia, %(ditto)
%urlcolor=Bittersweet,
urlcolor=RedOrange,
plainpages=false %(ditto)
]{hyperref}

\hypersetup{%
  pdftitle={Curriculum Vitae},
  pdfauthor={Karolina Krzyzanowska},
 % colorlinks={true},
 % linkcolor={pink}
}

\setlength{\parindent}{0pt}
\geometry{a4paper, lmargin=2cm, rmargin=2cm, bmargin=2cm, tmargin=2cm}
\begin{document}

\newcommand{\headone}[1] {
 \section*{#1}
\rule[2.5ex]{\textwidth}{2pt}}

\newcommand{\headtwo}[1]{
  \subsection*{#1}
  \rule[1.5ex]{\textwidth}{.5pt}}

\newcommand{\headthree}[1]{ {\large #1}}

\newcommand{\link}[1]{ \texttt{\href{#1}{#1}}}

\begin{small}

  \headone{Curriculum Vit\ae}
  
  Karolina Krzyżanowska\\
  \href{mailto:k.h.krzyzanowska@rug.nl}{\texttt{k.h.krzyzanowska@rug.nl}}\\
  Department of Theoretical Philosophy, Faculty of Philosophy, University of Groningen\\
  Oude Boteringestraat 52, 9712 GL Groningen, The Netherlands\\
  Homepage: \href{http://karolinakrzyzanowska.com}{\texttt{http://karolinakrzyzanowska.com}}\\
  Project's page:
  \href{http://formalphilosophy.org}{\texttt{http://formalphilosophy.org}}.

  \headtwo{Personal details:}
  \begin{itemize}
  \item Date of birth: April 4, 1985.
  \item Nationality: Polish.
  \end{itemize}

  
  \headtwo{Academic interests:}
  \begin{itemize}
  \item AOS: Philosophy of language, psychology of reasoning.
  \item AOC: Epistemology, philosophical logic, cognitive science,
    philosophy of mind.
  \end{itemize}
  
  \headtwo{Academic work experience:}
  \begin{itemize}
  \item PhD student at
    the %\href{http://formalphilosophy.org/}{Formal Epistemology Project}\\
    \emph{Department of Theoretical Philosophy, \href{http://www.rug.nl/filosofie/}{Faculty of Philosophy}\\
      \href{http://www.rug.nl}{University of Groningen}, The Netherlands}\\
    % \href{http://formalphilosophy.org/}{Formal Epistemology
    % Project}\\
    Supervisor: Prof. Igor Douven\\
    November 2010 --- Present.


  \item PhD student at the  \href{http://formalphilosophy.org/}{Formal Epistemology Project}\\
    \emph{Centre for Logic and Analytic Philosophy, \href{http://hiw.kuleuven.be/eng/}{Institute of Philosophy}\\
      \href{http://www.kuleuven.be/english}{Katholieke Universiteit Leuven}, Belgium}\\
    Supervisor: Prof. Igor Douven\\
    November 2009 --- October 2010.

  \end{itemize}

  % \headtwo{Non-academic work experience:}
  % \begin{itemize}
  % \item \emph{Customer service advisor at
  %   \href{http://www.plfon.pl/en/}{Peoplefone
  %   Polska SA}, Poland}\\
  %   November 2007 --- August 2008. \\
  % \end{itemize}


  
  \headtwo{Education:}
  \begin{itemize}
  \item \emph{\href{http://hiw.kuleuven.be/eng/}{Institute of Philosophy}, \href{http://www.kuleuven.be}{Katholieke Universiteit Leuven}, Belgium}\\
    Erasmus Exchange Programme\\
    September 2008 --- June 2009.
    
  \item \emph{\href{http://www.filozofia.uw.edu.pl}{Institute of
        Philosophy}, \href{http://www.uw.edu.pl}{University of Warsaw}, Poland}\\
    October 2004 --- September 2009\\
    Degree: M.A. in Philosophy\\
    Thesis: \emph{Belief Reports in Linguistic and Cognitive
      Perspective} (in Polish)\\
    Advisor: Dr Justyna Grudzińska.
    


  \item Summer schools: %and workshops:
    \href{http://www.esslli2012.pl}{ESSLLI 2012} in Opole, Poland;
      %%
    \href{http://esslli2011.ijs.si}{ESSLLI 2011} in Ljubljana,
    Slovenia;
      %%
    % Workshop on
    % \href{http://www.phil-fak.uni-duesseldorf.de/conditionals/}{Conditionals,
    % Counterfactuals and Causes in Uncertain
    % Environments} 2011 in Düsseldorf, Germany;
      %%
    ESSLLI 2010 in Copenhagen, Denmark;
      %%
    % \href{http://www.bbk.ac.uk/psychology/our-research/londonreasoningworkshop/lrw5}{5th
    % LRW} 2010 at Birkbeck College London, UK;
    %   %%
    % \href{http://homepages.ulb.ac.be/~uicm3/}{UICM III} 2010 in
    % Brussels, Belgium;
      %%
    Logic Workshop
    \href{http://www.logika.uw.edu.pl/warsztaty2007/index.html}{``Logic
      and Cognition''} 2007 on Szrenica, Szklarska Poręba, Poland.

  \item \emph{High School: Liceum Ogólnokształcące im. Bolesława
      Prusa, Skierniewice, Poland}\\
    September 2000 --- June 2004.

  \end{itemize}


  \headtwo{Published papers:}
  \begin{itemize}
    \item `Rethinking Gibbard's Riverboat Argument' with Sylvia
    Wenmackers and Igor Douven, \emph{Studia Logica}, DOI: 10.1007/s11225-013-9507-2, 2013.
    % \href{http://link.springer.com/article/10.1007/s11225-013-9507-2}

   \item `Inferential Conditionals and Evidentiality' with Sylvia
    Wenmackers and Igor Douven, \emph{Journal of Logic, Language and Information}, DOI: 10.1007/s10849-013-9178-4, 2013.
    % \href{http://link.springer.com/article/10.1007/s10849-013-9178-4}{}

  \item `Belief Ascription and the Ramsey Test', \emph{Synthese}
    Vol. 190, No. 1, 2013, pp. 21-36. % DOI:
    % \href{http://www.springerlink.com/content/6r04w05l4l457485/}{10.1007/s11229-012-0160-5}.

  \item `Conditionals, Inference, and Evidentiality' with Sylvia
    Wenmackers, Igor Douven and Sara Verbrugge, in: Jakub Szymanik and
    Rineke Verbrugge (eds.):
    \href{http://ceur-ws.org/Vol-883/}{`Proceedings of the Logic \&
      Cognition Workshop at ESSLLI 2012, Opole, Poland, 13-17 August,
      2012'}, vol. 883 of CEUR Workshop Proceedings, CEUR-WS.org,
    pp. 38-47, 2012.

  \item `Ambiguous Conditionals' in: Piotr Stalmaszczyk (ed.):
    `Philosophical and Formal Approaches to Linguistic Analysis',
    Ontos Verlag, pp. 315-332, 2012.

  \item `Sprawozdania z przekonań w perspektywie filozofii języka i
    kognitywistyki' (`Belief Reports in the perspective of philosphy
    of language and cognitive science'), \emph{Przegląd Filozoficzny}
    No. 3 (75), pp. 297-319, 2010.

  \end{itemize}

  \headtwo{Work in progres:}
  \begin{itemize}
   \item Conditionals in the context of deliberation.
   \item An experimental study on implicatures and their effect on people's acceptability, probability, and truth value judgments.
  \end{itemize}
  

  \headtwo{Invited talks:}
  \begin{itemize}
  \item \emph{Conditionals and Inferences} at Logic and Interactive
    Rationality (\href{http://www.illc.uva.nl/lgc/seminar/}{LIRa})
    Seminar at the Institute of Logic, Language, and Computation,
    University of Amsterdam, 7 March 2013.
  \item \emph{Inferential Conditionals} at
    \href{https://sites.google.com/site/jannekehuitink/workshop}{Conditionals
      Workshop}, Faculty of Philosophy, University of Groningen, 9
    November 2012.
  \end{itemize}
  
  \headtwo{Contributed talks:}
  \begin{itemize}

  \item \emph{The Variety of Conditionals and the Evidential Function
      of Epistemic Modals} at the London Reasoning Workshop, Birkbeck
    College London, 25-26 July 2013.

  \item \emph{Deliberationally Useless Conditionals} at the 21st
    Annual Meeting of the European Society for Philosophy and
    Psychology (\href{http://espp2013.com}{ESPP2013}), Granada, Spain,
    9-12 July 2013.

  \item \emph{Rethinking Gibbard’s Riverboat Argument} at the 8th
    Barcelona Workshop on Conditionals
    (\href{http://www.ub.edu/logosbw/bw8/index.html}{BW8}),
    Universitat de Barcelona, 26-28 June 2013.
    
  \item \emph{What ``Must'' and ``Should'' Can Mean} at the Graduate
    Conference in Theoretical Philosophy
    (\href{http://www.philos.rug.nl/GCTP2013/}{GCTP2013}), University
    of Groningen, 19 April
    2013. %18-20 April, http://www.philos.rug.nl/GCTP2013/
    
  \item \emph{Conditionals, Inference, and Evidentiality} at the
    \href{http://www.ai.rug.nl/SocialCognition/logic-cognition/}{Logic
      \& Cognition Workshop}, European Summer School for Logic,
    Language, and Information
    (\href{http://www.esslli2012.pl}{ESSLLI}), University of Opole, 15
    August
    2012. %13-17 August, http://www.ai.rug.nl/SocialCognition/logic-cognition/

  \item \emph{Ambiguity and Gibbard's Argument Against Propositional
      Theories of Conditionals} at the Second International Conference
    on Philosophy of Language and Linguistics
    (\href{http://ia.uni.lodz.pl/linguistics/events/philang-2011}{PhiLang2011}),
    Department of English and General Linguistics, University of Łódź,
    13 May
    2011. %12-14 May, http://ia.uni.lodz.pl/linguistics/events/philang-2011

  \item \emph{Belief Ascription and the Ramsey Test} at the 9th
    National Conference of the Italian Society for Analytic Philosophy
    (\href{http://www.filosofia.lettere.unipd.it/analitica/sifa2010/}{SIFA})
    ``Truth, Knowledge and Science'', Padua, 25 September
    2010. %23-25 September, http://www.filosofia.lettere.unipd.it/analitica/sifa2010/

    
  \item \emph{Belief Ascription and the Ramsey Test} at the
    International Congress of Italian Society for Logic and Philosophy
    of Science
    (\href{http://dinamico2.unibg.it/silfs/convegno2010.htm}{SILFS}),
    University of Bergamo, 16 December
    2010. %15-17 December, http://dinamico2.unibg.it/silfs/convegno2010.htm
  \end{itemize}
  

  \headtwo{Other presentations:}
  \begin{itemize}

\item Comment on Matteo Colombo's presentation \emph{Explanatory Reasoning, Moral Value, and Economic Incentive. Or\ldots How much is your moral judgment?}, 
    \href{http://www.psychologie.uni-freiburg.de/Members/singmann/operational2013}{Operationalization: An Interdisciplinary Workshop at the Edge of Experimental Psychology and Analytical Philosophy}, Freiburg, 16 October 2013.

  \item \emph{Inferential Conditionals and the Meaning of Some
      Epistemic Modals} (poster) at the Sixth Semantics and Philosophy
    in Europe colloquium
    (\href{http://spe6conference.wordpress.com}{SPE6}),
    Saint-Petersburg State University, 10-14 June 2013.
    
  \item \emph{Indicatives and a~Problem of a~Bad Advice} at the WiP
    Seminar, Faculty of Philosophy, University of Groningen, 25 March
    2013.
    
  \item \emph{What inferential conditionals can reveal about epistemic
      modals. An experimental study of evidentiality} (poster) at the
    3rd Workshop of Experimental Philosophy Group UK:
    \href{https://www.nottingham.ac.uk/philosophy/research/conferences/workshop-intuitions-experimentsandphilosophy.aspx}{Inuitions,
      Experiments, and Philosophy}, Nottingham, September 2012.

  \item \emph{Truth Conditions for Conditional Sentences, Allan
      Gibbard and Ambiguity} at the PCCP Seminar, Faculty of
    Philosphy, University of Gronigen, 22 September 2011.
    
  \item \emph{What is going on in the nearset possible world\ldots}
    (poster), Meeting of the European Society for Philosophy and
    Psychology
    (\href{http://www.ruhr-uni-bochum.de/philosophy/espp2010/index.html}{ESPP2010}),
    Bochum and Essen, August 2010.

  \item Comment on Benjamin Hoffmann's paper \emph{Belief Revision for
      Dynamic Reliability Ranks}, The Bi-Annual Konstanz--Leuven
    \href{http://formalphilosophy.org/node/580}{Workshop in Formal
      Epistemology}, Leuven, 28 January 2010.
  \end{itemize}

  



  % \headtwo{}
  % \begin{itemize}
  % \item Socrates-Erasmus stipend to
  %   \href{http://www.kuleuven.be}{Katholieke Universiteit Leuven},
  %   Belgium (ten month stipend for the academic year 2008 --- 2009).
  % \item Scholarship for academic performance in the academic year
  %   2007 --- 2008.
  % \end{itemize}
  
  
  \headtwo{Teaching:}
  \begin{itemize}
  \item Spring 2013: master's level course \emph{Philosophy of Logic:
      Conditionals}, together with I.~Douven, spring 2013.
  \end{itemize}

  
  \headtwo{Activities:}
  \begin{itemize}
  \item Co-organization of the
    \href{http://www.philos.rug.nl/GCTP2013/}{Graduate Conference in
      Theoretical Philosophy} (Groningen, 18-20 April 2013).
  \end{itemize}


  \headtwo{Miscellaneous:}
  \begin{itemize}
  \item Languages: Polish (mother tongue), English (fluent), Dutch
    (basic speaking, advanced reading), German (reading, Zentrale
    Mittelstufe Pr\"ufung 2006)
  \item Computer skills: \LaTeX$_{2e}$, MS Office and OpenOffice.org,
    (X)HTML/CSS, basics of the R environment for statistical
    computing.

  \item Hobbies: playing cello, painting, climbing.
  \end{itemize}


\end{small}
\end{document}
