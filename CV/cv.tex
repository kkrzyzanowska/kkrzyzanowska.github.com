%%% A template to produce a nice-looking Curriculum Vitae.
%%% Kieran Healy <kjhealy@gmail.com>
%%% Most recent version is at http://kjhealy.github.com/kjh-vita
%%%
%%% ------------------------------------------------------------------------
%%% Requirements that are included in a modern tex distribution:
%%% ------------------------------------------------------------------------
%%% xelatex
%%% fontspec.sty
%%% hyperrref.sty
%%% xunicode.sty
%%% color.sty
%%% url.sty
%%% fancyhdr.sty
%%% memoir.cls
%%% fontawesome.sty
%%%
%%% 
%%% 
%%% ------------------------------------------------------------------------
%%% Requirements from https://github.com/kjhealy/latex-custom-kjh
%%% ------------------------------------------------------------------------
%%% org-preamble-xelatex.sty
%%% memoir-article-styles.sty
%%%
%%% ------------------------------------------------------------------------
%%% Optional
%%% ------------------------------------------------------------------------
%%% git
%%% vc.sty
%%% revnum.sty
%%% Fonts
%%%
%%% ------------------------------------------------------------------------
%%% Note
%%%------------------------------------------------------------------------
%%% Because this is a hand-tweaked file, be on the look out for \medksip, 
%%% \bigskip and \newpage commands here and there, which are used to balance
%%% the layout or avoid widows & orphans, etc. You should of course add or 
%%% remove these as needed.
%%%------------------------------------------------------------------------

\documentclass[11pt,article,oneside]{memoir}   
\usepackage{org-preamble-xelatex} 
\usepackage{fontawesome,url}


%%%------------------------------------------------------------------------
%%% Metadata
%%%------------------------------------------------------------------------

%% Change as needed. Or just add me as a coauthor. Only some of these are 
%% used below in the hyperref declaration and address banner section.
\def\myauthor{Karolina Krzy\.zanowska}
\def\mytitle{Vita}
\def\mycopyright{\myauthor}
\def\mykeywords{}
\def\mybibliostyle{plain}
\def\mybibliocommand{}
\def\mysubtitle{}
\def\myaffiliation{}
\def\myaddress{}
\def\myemail{k.krzyzanowska@lmu.de}
\def\myweb{http://karolinakrzyzanowska.com}
\def\myphone{}
\def\myfax{}
\def\mytwitter{}
\def\myversion{}
\def\myrevision{}


\def\myaffiliation{Center for Advanced Studies \& Munich Center for Mathematical Philosophy}
\def\myauthor{Karolina Krzy\.zanowska}
\date{} % not used (revision control instead)
\def\mykeywords{}

%%%------------------------------------------------------------------------
%%% Document
%%%------------------------------------------------------------------------
\begin{document}

%% Choose fonts for use with xelatex
%% Minion and Myriad are widely available, from Adobe. 
%% Pragmata is available to buy at http://www.fsd.it/fonts/pragma.htm
%% and is worth every penny. Any good monospace font will work fine, though.
%% Consolas or inconsolata are good alternatives.
\setromanfont[Mapping={tex-text}, 
	Numbers={OldStyle},
	Ligatures={Common}]{Minion Pro}
\setsansfont[Mapping=tex-text,
	Ligatures={Common}, 
	Colour=AA0000]{Unit-Medium}
\setmonofont[Mapping=tex-text,Scale=0.72]{Inconsolata} 

\newfontface\scheader[SmallCapsFont={Minion Pro},SmallCapsFeatures={Letters=SmallCaps}]{Minion Pro}

\newfontface\addressblock[Mapping={tex-text}, 
	Numbers={OldStyle},
	Ligatures={Common}]{Minion Pro Medium}


%%%------------------------------------------------------------------------
%%% Local commands
%%%------------------------------------------------------------------------

%% Marginal header
%% Note: as the document goes on you may need to introduce a (gradually increasing)
%% \vspace element to keep the marginal header pleasingly aligned with the first 
%% item in the body text. Like this: \marginhead{{\vskip 0.4em}Grants}, or 
%% \marginhead{{\vskip 0.8em}Service}. Experiment as needed.
\newcommand{\marginhead}[1]{\marginpar{\textsf{{\footnotesize\vspace{-1em}\flushright #1}}}}


%% [optional] custom ampersand (font consistent with the one chosen above)
% \newcommand{\amper}{{\fontspec[Scale=.95,Colour=AA0000]{Minion Pro Medium}\selectfont\&\,}}

%% No bullets on labels
\renewcommand{\labelitemi}{~} 

%% Custom hanging indent for vita items
\def\ind{\hangindent=1 true cm\hangafter=1 \noindent}
%\def\ind{\hangindent=18pt\hangafter=1 \noindent}
\def\labelitemi{~}
\renewcommand{\labelitemii}{~}

%%%------------------------------------------------------------------------
%%% Page layout
%%%------------------------------------------------------------------------

% These lines will insert git revision info in the footer, using the vc
% package---see docs for vc package for details. Comment out this line
% if you're not using vc, and also remove the \input{vc} line above.
\pagestyle{kjh}
\thispagestyle{kjhgit}


%%%------------------------------------------------------------------------
%%% Address and contact block
%%%------------------------------------------------------------------------
% \begin{minipage}[t]{2.95in}
%  \flushright {\footnotesize 
% %CAS, LMU Munich\\
% Geschwister-Scholl-Platz 1\\ \vspace{-0.05in} 80539 Munich, Germany.}  
%  % Geschwister-Scholl-Platz 1\\ 80539 Munich,\\ \vspace{-0.05in} Germany.}  
  
% \end{minipage}
\hfill     
%\begin{minipage}[t]{0.0in}
% dummy (needed here)
%\end{minipage}
\hfill
\begin{minipage}[t]{1.8in}
  \flushright \footnotesize  \addressblock %\\ 
  {\small  \texttt{\href{mailto:\myemail}{\myemail}} \, \faEnvelope} \\
  {\small  \texttt{\href{\myweb}{\myweb}} \, \faGlobe}
\end{minipage}

% \begin{minipage}[t]{1.8in}
%   \flushright \footnotesize  \addressblock \myphone \, \faPhone \\ 
%   {\small  \texttt{\href{mailto:\myemail}{\myemail}} \, \faEnvelope} \\
%   {\small  \texttt{\href{\myweb}{\myweb}} \, \faGlobe}
% \end{minipage}

\medskip

%% Name 
\noindent{\LARGE\scheader \textsc{karolina krzy\.zanowska}}
\reversemarginpar

\bigskip       


%% Appointments
% \medskip
\marginhead{\sffamily academic appointments}

\ind October 2016 --- Present\\
Postdoctoral Research Fellow at the \href{http://www.en.cas.uni-muenchen.de/rir/senior_rir/current_senior_rir/hartmann_stephan/index.html}{Scientific Reasoning and Argumentation Project}, Center for Advanced Studies, LMU Munich. \\
Principal Investigator: Prof.\ Dr.\ Stephan Hartmann

\ind October 2014 --- September 2016\\
Postdoctoral Research Fellow at the Munich Center for Mathematical Philosophy, LMU Munich, Germany.
%\href{http://www.en.uni-muenchen.de/}{Ludwig-Maximilian-Universität München}
%\href{http://www.mcmp.philosophie.uni-muenchen.de/}{Munich Center for Mathematical Philosophy}

\ind June 2014 --- September 2014\\ 
    % in the \href{http://www.uib.no/en/rg/lii}{Logic, Information and Interaction} group 
    Visiting researcher at the Department of Information Science and Media Studies, University of Bergen, Norway. 
    % \emph{\href{http://www.uib.no/en}{University of Bergen}, Norway

\ind February 2014 --- July 2014\\
	Doctoral researcher in the Project PF 740/2-1 ``Rational reasoning with conditionals and probabilities. Logical foundations and empirical evaluations'' within the DFG Priority Program SPP 1516 ``New Frameworks of Rationality.'' Munich Center for Mathematical Philosophy, LMU Munich, Germany. \\
  Principal Investigator: Dr. Dr. Niki Pfeifer
% \href{http://www.pfeifer-research.de/spp.html}{Project PF 740/2-1}
    
\ind November 2010 --- October 2013\\
	PhD student at the Department of Theoretical Philosophy, Faculty of Philosophy, University of Groningen, The Netherlands. \\
    % \href{http://www.rug.nl/filosofie/}{Faculty of Philosophy}\\
    % \href{http://www.rug.nl}{University of Groningen}, The Netherlands\\
    % \href{http://formalphilosophy.org/}{Formal Epistemology
    % Project}\\	
     Supervisor: Prof. Igor Douven

\ind November 2009 --- October 2010\\ 
    PhD student at the  \href{http://formalphilosophy.org/}{Formal Epistemology Project}, Centre for Logic and Analytic Philosophy, Institute of Philosophy, University of Leuven, Belgium.\\ 
    % \emph{Centre for Logic and Analytic Philosophy, \href{http://hiw.kuleuven.be/eng/}{Institute of Philosophy}\\
    % \href{http://www.kuleuven.be/english}{Katholieke Universiteit Leuven}, Belgium}\\
     Supervisor: Prof. Igor Douven



\bigskip

%% Education

\marginhead{\sffamily {{\vskip -0.23em} education}}

\ind Ph.D., Philosophy, University of Groningen, the Netherlands, February
2015.

\ind \hspace{0.35in} \footnotesize 
Thesis: \emph{\href{http://karolinakrzyzanowska.com/pdfs/krzyzanowska-phd-final.pdf}{Between ``If'' and ``Then.'' Towards an Empirically Informed Philosophy of Conditionals}}. \\
Advisor: \href{https://www.researchgate.net/profile/Igor_Douven}{Prof. Igor Douven}\\
Committee: \href{https://www.dur.ac.uk/psychology/staff/?id=4610}{Prof. David Over}, \href{http://www.uni-regensburg.de/philosophie-kunst-geschichte-gesellschaft/theoretische-philosophie/personen/prof-dr-hans-rott/}{Prof. Hans Rott}, \href{http://www.rinekeverbrugge.nl}{Prof. Rineke Verbrugge}. 
\normalsize 
\vspace{0.05in}

\ind M.A., Philosophy, University of Warsaw, Poland, September 2009. 

\ind \hspace{0.35in} \footnotesize Thesis (in Polish): \emph{Belief Reports in Linguistic and Cognitive Perspective}. \\
Advisor: Dr. Justyna Grudzińska. 
\normalsize \vspace{0.01in}
  %  \href{http://www.uw.edu.pl}{University of Warsaw}, Poland\\
  % October 2004 --- September 2009\\
    

\ind Erasmus Exchange Programme, University of Leuven, 2008 --- 2009.\\ 
    % \href{http://www.kuleuven.be}{Katholieke Universiteit Leuven}, Belgium\\
    \ind \hspace{0.35in} \footnotesize 
    Courses: Philosophy, Cognitive Science, Artificial Intelligence.
	\normalsize \vspace{0.01in}

<<<<<<< HEAD
\ind Summer schools: \href{https://irsi2016.de}{IRSI 2016}, \href{http://www.esslli2012.pl}{ESSLLI 2012}, \href{http://esslli2011.ijs.si}{ESSLLI 2011},  ESSLLI 2010, \href{http://www.logika.uw.edu.pl/warsztaty2007/}{Warsztaty Logiczne 2007 ``Logika i Kognitywistyka.''}

%%%%%%%%%%%%% %%%%%%%%%%%%% %%%%%%%%%%%%% %%%%%%%%%%%%%
% Research visits:
%  Birkbeck College London, March 2015. Host: Prof. Ulrike Hahn
%%%%%%%%%%%%% %%%%%%%%%%%%% %%%%%%%%%%%%% %%%%%%%%%%%%%

\bigskip
\bigskip
=======
  
  \headtwo{Education:}
  \begin{itemize}
    \item PhD student at
    the %\href{http://formalphilosophy.org/}{Formal Epistemology Project}\\
    \emph{Department of Theoretical Philosophy, \href{http://www.rug.nl/filosofie/}{Faculty of Philosophy}\\
    \href{http://www.rug.nl}{University of Groningen}, The Netherlands}\\
    % \href{http://formalphilosophy.org/}{Formal Epistemology
    % Project}\\
    Degree: Ph.D.\ in Philosophy\\
    Thesis: \emph{\href{http://karolinakrzyzanowska.com/pdfs/krzyzanowska-phd-final.pdf}{Between ``if'' and ``then.'' Towards an empirically informed philosophy of conditionals.}} 
    November 2010 --- February 2015.\\
    Supervisor: Prof.\ Igor Douven  
    
    \item \emph{\href{http://www.filozofia.uw.edu.pl}{Institute of Philosophy}, \href{http://www.uw.edu.pl}{University of Warsaw}, Poland}\\
    October 2004 --- September 2009\\
    Degree: M.A.\ in Philosophy\\
    Thesis: \emph{Belief Reports in Linguistic and Cognitive Perspective} (in Polish)\\
    Advisor: Dr.\ Justyna Grudzińska.
    
    \item \emph{\href{http://hiw.kuleuven.be/eng/}{Institute of Philosophy}, \href{http://www.kuleuven.be}{Katholieke Universiteit Leuven}, Belgium}\\
    Erasmus Exchange Programme\\
    September 2008 --- June 2009.
  

    \item Summer schools and workshops:
    \href{http://www.esslli2012.pl}{ESSLLI 2012} in Opole, Poland;
      %%
    \href{http://esslli2011.ijs.si}{ESSLLI 2011} in Ljubljana, Slovenia;
      %%
    % Workshop on
    % \href{http://www.phil-fak.uni-duesseldorf.de/conditionals/}{Conditionals,
    % Counterfactuals and Causes in Uncertain
    % Environments} 2011 in Düsseldorf, Germany;
      %%
    ESSLLI 2010 in Copenhagen, Denmark;
      %%
    % \href{http://www.bbk.ac.uk/psychology/our-research/londonreasoningworkshop/lrw5}{5th
    % LRW} 2010 at Birkbeck College London, UK;
    %   %%
    % \href{http://homepages.ulb.ac.be/~uicm3/}{UICM III} 2010 in
    % Brussels, Belgium;
      %%
    Logic Workshop \href{http://www.logika.uw.edu.pl/warsztaty2007/index.html}{``Logic
      and Cognition''} 2007 on Szrenica, Szklarska Poręba, Poland.

    % \item \emph{High School: Liceum Ogólnokształcące im. Bolesława Prusa, Skierniewice, Poland}\\
    % September 2000 --- June 2004.

  \end{itemize}


  \headtwo{Published papers:}
  \begin{itemize}
    \item `Rethinking Gibbard's Riverboat Argument' with Sylvia Wenmackers and Igor Douven, \emph{Studia Logica}, Vol. 102, No. 4, pp. 771-792, 2014.
    % \href{http://link.springer.com/article/10.1007/s11225-013-9507-2}

    \item `Inferential Conditionals and Evidentiality' with Sylvia Wenmackers and Igor Douven, \emph{Journal of Logic, Language and Information}, Vol.~22 No.~3, pp.~315–334, 2013.
    % \href{http://link.springer.com/article/10.1007/s10849-013-9178-4}{}

    \item `Belief Ascription and the Ramsey Test', \emph{Synthese} Vol.~190, No.~1, pp.~21-36, 2013. % DOI:
    % \href{http://www.springerlink.com/content/6r04w05l4l457485/}{10.1007/s11229-012-0160-5}.

    \item `Conditionals, Inference, and Evidentiality' with Sylvia Wenmackers, Igor Douven and Sara Verbrugge, in: Jakub Szymanik and Rineke Verbrugge (eds.): \href{http://ceur-ws.org/Vol-883/}{`Proceedings of the Logic \& Cognition Workshop at ESSLLI 2012, Opole, Poland, 13-17 August, 2012'}, vol.~883 of CEUR Workshop Proceedings, CEUR-WS.org, pp.~38-47, 2012.

    \item `Ambiguous Conditionals' in: Piotr Stalmaszczyk (ed.): `Philosophical and Formal Approaches to Linguistic Analysis', Ontos Verlag, pp.~315-332, 2012.

    \item `Sprawozdania z przekonań w perspektywie filozofii języka i kognitywistyki' (`Belief Reports in the perspective of philosphy of language and cognitive science'), \emph{Przegląd Filozoficzny} No.~3 (75), pp.~297-319, 2010.

  \end{itemize}

  \headtwo{Forthcoming papers:}
  \begin{itemize}
   \item `Deliberationally useless conditionals,' manuscript under revision.
    
    \begin{description}{\footnotesize
      \item[Abstract:] \textit{Decision theorists tend to treat indicative conditionals with reservation, as they can be sources of bad or even conflicting advice. Contrary to the mainstream view, Keith DeRose argued that indicatives can be conditionals of deliberation, though some of them are unassertable in the contexts of de- liberation. However, as we are going to argue, DeRose failed to recognise the reason why this is the case. We will show that a recently proposed ``inferential'' account of indicative conditionals provides a more straightforward explanation of this phenomenon and, simultaneously, a way of telling ``deliberationally useless'' and ``deliberationally useful'' conditionals apart.}}
    \end{description}

   
   \item `Antecedents, consequents and something in between,' in preparation.
    \begin{description}{\footnotesize
      \item[Abstract:] \textit{It is a common intuition that the antecedent of an indicative conditional should have something to do with its consequent, that they should be somehow connected. However, only very few semantic theories of conditionals do justice to this intuition and the majority tends to relegate it to the realm of pragmatics. Yet no one has offered a full-fledged pragmatic account of the oddness of missing-link conditionals. The aim of this paper is to discuss possible pragmatic explanations of the phenomenon and, consequently, to show how they fail.}}
    \end{description}

  \item `Approximate interpretation of numerals: number sense and semantics/pragmatics interface,' in preparation (with Paula Quinon).
    \begin{description}{\footnotesize
      \item[Abstract:] \textit{It is a well known fact that expressions like ``5 million,'' ``two hundreds'' or even, in some contexts, numerals denoting smaller numbers like ``forty'' tend to be interpreted as approximations. Although the pragmatic aspects of this phenomenon has been widely discussed in the linguistic literature, the semantics of sentences containing round numbers has not attracted much attention. Moreover, it is far from obvious what is the truth value of a sentence like ``Norway has a population of 5 million,'' even if we know that the exact number of the country's inhabitants is 5,109,059. Can a sentence be assertable yet false? Drawing from recent developments in cognitive science, we argue that our preference for vague interpretation of numerals is due to the approximate number system being the primary source of our mental representation of numbers. Furthermore, we offer a conceptual spaces model of the interpretation of round numbers.}}
    \end{description}



   \item `Persuading with conditionals,' in preparation (with Ulrike Hahn and Peter Collins).
   % \item Empirical evaluation of different solutions to the old evidence problem (with Stephan Hartmann).
   \item An experimental study on implicatures and their effect on people's assertability, believability, and truth value judgments (with Igor Douven and Henrik Singmann).
   \item `What psychology of reasoning can and cannot tell us about the semantics of conditionals?' in preparation.  
  \end{itemize}
  

  \headtwo{Invited talks:}
  \begin{itemize} 
  	\item \emph{Persuading with conditionals} at the symposium \emph{Dynamic inference and belief revision}  organised by Mike Oaksford at the \href{http://sites.clps.brown.edu/ict2016/}{International Conference on Thinking (ICT2016)}, Providence, RI, 4-6 August 2016. 


    \item \emph{What psychology of reasoning can teach us about the meaning of indicative conditionals: The case of a metalinguistic theory} at the \href{http://www.f.bg.ac.rs/fil-konf/konferencija2-home.html}{2nd Belgrade Conference on Conditionals}, Belgrade, Serbia, 28-31  May 2016.

    \item Keynote talk: \emph{What Psychology of Reasoning Can Tell Us About the Meaning of Indicative Conditionals} at the \href{http://logicalconnectives.uw.edu.pl/}{Warsaw Workshop in Philosophy of Language: Logic and Meaning}, Warsaw, Poland, 13-14 May 2016.

    \item \emph{What are indicative conditionals about?} at EuNoC Workshop, Leeds, UK, 19-20 January 2016.
>>>>>>> origin/master

%% AOS and AOC
\marginhead{\sffamily {{\vskip -0.23em} research profile}}

<<<<<<< HEAD
\ind AOS: philosophy of language, psychology of reasoning, philosophy of psychology.

\ind AOC: philosophy of mind, epistemology, philosophical logic.
 %% add a note on what my research is?

 \bigskip

%% Publications
\marginhead{\sffamily {\vskip -0.25em} publications}

\ind Karolina Krzyżanowska, Sylvia Wenmackers and Igor Douven,
\href{http://link.springer.com/article/10.1007/s11225-013-9507-2}{``Rethinking Gibbard's Riverboat Argument''}, \emph{Studia Logica}, Vol. 102, No. 4, pp. 771-792, 2014.

\ind Karolina Krzyżanowska, Sylvia Wenmackers and Igor Douven,
\href{http://link.springer.com/article/10.1007/s10849-013-9178-4}{``Inferential Conditionals and Evidentiality''}, 
 \emph{Journal of Logic, Language and Information}, Vol.~22 No.~3, pp.~315–334, 2013.

\ind Karolina Krzyżanowska, \href{http://link.springer.com/article/10.1007%2Fs11229-012-0160-5}{``Belief Ascription and the Ramsey Test''}, \emph{Synthese} Vol.~190, No.~1, pp.~21-36, 2013. 

\ind Karolina Krzyżanowska, Sylvia Wenmackers, Igor Douven and Sara Verbrugge \href{http://ceur-ws.org/Vol-883/paper4.pdf}{``Conditionals, Inference, and Evidentiality''} in: Jakub Szymanik and Rineke Verbrugge (eds.): \href{http://ceur-ws.org/Vol-883/}{`Proceedings of the Logic \& Cognition Workshop at ESSLLI 2012, Opole, Poland, 13-17 August, 2012'}, vol.~883 of CEUR Workshop Proceedings, CEUR-WS.org, pp.~38-47, 2012.

\ind Karolina Krzyżanowska ``Ambiguous Conditionals'' in: Piotr Stalmaszczyk (ed.): \emph{Philosophical and Formal Approaches to Linguistic Analysis}, Ontos Verlag, pp.~315-332, 2012.

 \ind Karolina Krzyżanowska ``Sprawozdania z przekonań w perspektywie filozofii języka i kognitywistyki'' (``Belief Reports in the perspective of philosphy of language and cognitive science''), \emph{Przegląd Filozoficzny} No.~3 (75), pp.~297-319, 2010.

\bigskip 

% {\vskip -0.25em}
\marginhead{\sffamily working papers}
\ind ``Deliberationally Useless Conditionals,'' manuscript under revision.

\ind \hspace{0.35in} \footnotesize 
Abstract: \textit{Decision theorists tend to treat indicative conditionals with reservation, since they can be sources of bad or even conflicting advice. Nevertheless, many indicatives are clearly useful in contexts of deliberation, and denying them all a role in such contexts seems to be an overkill. This paper shows that a recently revived inferential view on conditionals provides a straightforward explanation of why some indicatives are unassertable in contexts of deliberation, and, consequently, it provides us with a way of telling ``deliberationally useless'' and ``deliberationally useful'' conditionals apart.}
\normalsize 
\vspace{0.05in}

\ind ``Conditionals and Testimony'' with Peter Collins, Stephan Hartmann, Gregory Wheeler, and Ulrike Hahn.

\ind \hspace{0.35in} \footnotesize 
Abstract: \textit{In this paper, we present some basic phenomena involved in responding to conditional utterances. In particular, we provide a brief summary of a number of experiments examining belief change in response to hearing a conditional. We then examine the implications of these phenomena for theories of the conditional and conditional reasoning. Finally, we present a model against which these phenomena might be understood. }
\normalsize 
\vspace{0.05in}

\ind ``Probabilistic Relevance vs. Discourse Coherence Relations'' with Ulrike Hahn and Peter Collins.

\ind \hspace{0.35in} \footnotesize 
Abstract: \textit{We explore different factors influencing the assertability of indicative conditionals. In particular, we present an experimental study aiming at disentangling the discourse coherence relations that must connect any two consecutive elements of discourse, and a stronger relation of probabilistic relevance that, as we shall argue, must be present for a conditional to be assertable.}
\normalsize 
\vspace{0.05in}


\ind ``Assertions about Numbers'' with Paula Quinon:

\ind \hspace{0.35in} \footnotesize 
Abstract: \textit{It is a well known fact that expressions like ``5 million'' or ``two hundreds'' tend to be interpreted as approximations. Although the pragmatics of rounding up has been widely discussed in linguistic literature, semantic consequences of this phenomenon have not attracted that much attention. Drawing from the recent developments in cognitive science, and from research on the approximate number sense in particular, we discuss possible truth conditions for sentences involving rounded up numerals.}
\normalsize 
\vspace{0.05in}

\ind ``The Semantics-Pragmatics Interface: An Empirical Investigation'' with Igor Douven:

\ind \hspace{0.35in} \footnotesize 
Abstract: \textit{Linguists and philosophers commonly distinguish between semantics and pragmatics, where the former concerns the truth or falsity of linguistic items and the latter concerns aspects of the usage of such items that may make them unassertable even when true. Common though the distinction is, there is an ongoing controversy about where exactly the line between semantics and pragmatics is to drawn. In this paper, we report two experiments meant to investigate empirically whether there is any pre-theoretic distinction that might help settle the debate. }
% The same experiments are meant to shed light on a related question, namely, whether pragmatic aspects of language use pertain only at the level of assertability and not at that of believability. Our results suggest that ordinary people do not reliably distinguish between truth, assertability, or believability. We argue that this has consequences for the methodology of experimental semantics and pragmatics.}
\normalsize 
\vspace{0.05in}


\ind ``The Limits of a Pragmatic Explanation: the Case of Indicative Conditionals.''

\ind \hspace{0.35in} \footnotesize 
Abstract: \textit{It is a common intuition that the antecedent of an indicative conditional should have something to do with its consequent, that they should be somehow connected. However, only very few semantic theories of conditionals do justice to this intuition and the majority tends to relegate it to the realm of pragmatics. Yet no one has offered a full-fledged pragmatic account of the oddness of missing-link conditionals. The aim of this paper is to discuss possible pragmatic explanations of the phenomenon and, consequently, to show how they all fail.}
\normalsize 
\vspace{0.05in}



\ind ``Content Matters: An Experimental Study of Conditionals and Conjunctions'' with Ulrike Hahn, and Peter Collins.

\ind \hspace{0.35in} \footnotesize 
Abstract: \textit{An experimental study on the assertability of conditionals and conjunctions with true constituent clauses.}
\normalsize 
\vspace{0.05in}



\bigskip 

\marginhead{\sffamily {\vskip 0.2em}conferences and workshops}
\medskip
    \ind Invited talk ``Indicative Conditionals: From Language to Reasoning and Back,'' \href{http://www.mcmp.philosophie.uni-muenchen.de/events/workshops/container/relevance_of_logic/index.html}{The Relevance of Logic to Human Reasoning} Workshop, Munich, Germany, November 2016. 

  	\ind ``Researching Missing-Link Conditionals: Methodological Obstacles and How (We Could at Least Try) to Avoid Them,'' Young Scientist's Forum of the \href{https://irsi2016.de}{International Rationality Summer Institute}, Aurich, Germany, September 2016. 
  	% 4-16 September 2016. 

    \ind ``Indicative Conditionals and the Search for the Semantics-Pragmatics Distinction'' \href{http://sites.clps.brown.edu/ict2016/}{International Conference on Thinking (ICT2016)}, Providence, RI, United States, August 2016.   
    % 4-6 August 2016. 

	\ind ``Persuading with Conditionals'', invited contribution to the Symposium \emph{Dynamic Inference and Belief Revision}  organised by Mike Oaksford at the \href{http://sites.clps.brown.edu/ict2016/}{International Conference on Thinking (ICT2016)}, Providence, RI, United States, August 2016.
	%4-6 August 2016. Invited

	\ind Comment on Ben Schwan and Reuben Stern's paper: ``A Causal Understanding of When and When Not to Jeffrey Conditionalize,'' \href{http://www.philos.rug.nl/few2016/}{Formal Epistemology Workshop 2016}, Groningen, June 2016.
  	% 20 June 2016.

    \ind Invited talk: ``What Psychology of Reasoning can Teach us about the Meaning of Indicative Conditionals: The Case of a Metalinguistic Theory,'' \href{http://www.f.bg.ac.rs/fil-konf/konferencija2-home.html}{2nd Belgrade Conference on Conditionals}, Belgrade, Serbia, May 2016.
    % 28-31  May 2016.

    \ind Invited keynote talk: ``What Psychology of Reasoning Can Tell Us About the Meaning of Indicative Conditionals,'' \href{http://logicalconnectives.uw.edu.pl/}{Warsaw Workshop in Philosophy of Language: Logic and Meaning}, Warsaw, Poland, May 2016. 
    % 13-14 May 2016.

    \ind Invited talk: ``What are Indicative Conditionals About?'' at EuNoC Workshop, Leeds, UK, January 2016.
    % 19-20 January 2016.

    \ind Invited talk: ``A New Kind of a Metalinguistic Theory of Conditionals'' at the workshop \emph{\href{http://cms.uni-konstanz.de/what-if/events/workshop-in-osnabrueck-september-2015-p1-and-p7/}{Do Conditionals have Truth Conditions? Dynamic and Pragmatic Aspects of Conditionals}}, GAP.9 congress, Osnabrück, Germany, September 2015.
    % 18-19 September 2015.

    \ind ``Between \emph{If} and \emph{Then}. What Do We Learn When We Learn a Conditional?'' at the Symposium \emph{If, Then, Otherwise: A Symposium on Conditionals}. 23rd Annual Meeting of the European Society for Philosophy and Psychology (\href{http://espp2015.ut.ee}{ESPP2015}), Tartu, Estonia, July 2015.
    % 14-17 July 2015.

    \ind ``The Ramsey Test and Conditionals of a Bad Advice'' at the Annual Meeting of the Priority Program \emph{New Frameworks of Rationality} (\href{http://www.spp1516.de}{SPP1516}), Etelsen, Germany, March 2015.
    % 15-18 March 2015.

    \ind ``Exact Numerals as Vague Quantifiers'' at the \href{https://sites.google.com/site/szklarskaporebaworkshop16/}{16th Szklarska Poręba Workshop} on the Roots of Pragmasemantics, Szrenica, Szklarska Poręba, Poland,  February 2015.
    % 20-23 February 2015.
=======
    \item \emph{What wrong with the Ramsey Test and why we should care} at the \href{http://www.mcmp.philosophie.uni-muenchen.de/events/weekly_talks_new/index.html}{MCMP Colloquium in Philosophy, Logic and Philosophy of Science}, LMU Munich, Germany, 13 May 2015.

    \item \emph{Between ``if'' and ``then.'' On what is wrong with the Ramsey Test and how to fix it} at the Reasoning and Argumentation Lab (\href{http://www.bbk.ac.uk/psychology/ral}{ReAL}) Seminar, Birkbeck College London, UK, 9 March 2015.

    \item \emph{Conditionals and inferences} at Logic and Interactive
    Rationality (\href{http://www.illc.uva.nl/lgc/seminar/}{LIRa})
    Seminar at the Institute of Logic, Language, and Computation,
    University of Amsterdam, The Netherlands, 7 March 2013.
  
    \item \emph{Inferential Conditionals} at
    \href{https://sites.google.com/site/jannekehuitink/workshop}{Conditionals Workshop}, Faculty of Philosophy, University of Groningen, The Netherlands, 9 November 2012.
  \end{itemize}
  
  \headtwo{Contributed talks:}
  \begin{itemize}
  	\item \emph{Researching Missing-Link Conditionals: Methodological Obstacles and how (We Could at Least Try) to Avoid Them} at the Young Scientisci Forum of the \href{https://irsi2016.de}{International Rationality Summer Institute}, Aurich, Germany, 4-16 September 2016. 

    \item \emph{Indicative conditionals and the search for the semantics-pragmatics distinction} at the \href{http://sites.clps.brown.edu/ict2016/}{International Conference on Thinking (ICT2016)}, Providence, RI, United States, 4-6 August 2016. 

    \item \emph{Between ``if'' and ``then.'' What do we learn when we learn a conditional?} at the Symposium \emph{If, Then, Otherwise: A Symposium on Conditionals}. 23rd Annual Meeting of the European Society for Philosophy and Psychology (\href{http://espp2015.ut.ee}{ESPP2015}), Tartu, Estonia, 14-17 July 2015.
>>>>>>> origin/master

	\ind Comment on Matteo Colombo's project: ``Explanatory Reasoning, Moral Value, and Economic Incentive. Or\ldots How much is your moral judgment?'' 
    \href{http://www.psychologie.uni-freiburg.de/Members/singmann/operational2013}{Operationalization: An Interdisciplinary Workshop at the Edge of Experimental Psychology and Analytical Philosophy}, Freiburg, October 2013.
    % 16 October 2013.

    \ind ``The Variety of Conditionals and the Evidential Function of Epistemic Modals'' at The 7th London Reasoning Workshop (\href{http://www.bbk.ac.uk/psychology/about-us/events/the-7th-london-reasoning-workshop}{LRW}), Birkbeck College London, United Kingdom, July 2013.
    % 25-26 July 2013.

<<<<<<< HEAD
    \ind ``Deliberationally Useless Conditionals'' at the 21st Annual Meeting of the European Society for Philosophy and Psychology (\href{http://espp2013.com}{ESPP2013}), Granada, Spain, July 2013.
    % 9-12 July 2013.
=======
    \item \emph{The Variety of Conditionals and the Evidential Function of Epistemic Modals} at The 7th London Reasoning Workshop (\href{http://www.bbk.ac.uk/psychology/about-us/events/the-7th-london-reasoning-workshop}{LRW}), Birkbeck College London, United Kingdom, 25-26 July 2013.
>>>>>>> origin/master

    \ind ``Rethinking Gibbard’s Riverboat Argument'' at the 8th Barcelona Workshop on Conditionals (\href{http://www.ub.edu/logosbw/bw8/index.html}{BW8}), Universitat de Barcelona, Spain, June 2013.
    % 26-28 June 2013.

	\ind Poster: ``Inferential Conditionals and the Meaning of Some  Epistemic Modals,'' Sixth Semantics and Philosophy in Europe Colloquium (\href{http://spe6conference.wordpress.com}{SPE6}), Saint-Petersburg State University, June 2013.
	% 10-14 June 2013.
  
    
    \ind ``What \emph{Must} and \emph{Should} Can Mean'' at the Graduate Conference in Theoretical Philosophy (\href{http://www.philos.rug.nl/GCTP2013/}{GCTP2013}), Groningen, The Netherlands, April 2013. 
    % 19 April 2013. 
    %18-20 April, http://www.philos.rug.nl/GCTP2013/
    

    \ind Invited talk: ``Inferential Conditionals'' at \href{https://sites.google.com/site/jannekehuitink/workshop}{Conditionals Workshop}, Faculty of Philosophy, University of Groningen, The Netherlands, November 2012.
 	% 9 November 2012.

	\ind Poster: ``What Inferential Conditionals Can Reveal About Epistemic Modals: An Experimental Study Of Evidentiality,'' 3rd Workshop of Experimental Philosophy Group UK: \href{https://www.nottingham.ac.uk/philosophy/research/conferences/workshop-intuitions-experimentsandphilosophy.aspx}{Inuitions, Experiments, and Philosophy}, Nottingham, September 2012.

    \ind ``Conditionals, Inference, and Evidentiality'' at the \href{http://www.ai.rug.nl/SocialCognition/logic-cognition/}{Logic \& Cognition Workshop}, European Summer School for Logic, Language, and Information (\href{http://www.esslli2012.pl}{ESSLLI}), Opole, Poland, August 2012. 
    % 15 August 2012. 
    %13-17 August, http://www.ai.rug.nl/SocialCognition/logic-cognition/

    \ind ``Ambiguity and Gibbard's Argument Against Propositional Theories of Conditionals'' at the Second International Conference on Philosophy of Language and Linguistics (\href{http://ia.uni.lodz.pl/linguistics/events/philang-2011}{PhiLang2011}),  Department of English and General Linguistics, Łódź, Poland, May 2011. 
    % 13 May 2011. 
    %12-14 May, http://ia.uni.lodz.pl/linguistics/events/philang-2011

    \ind ``Belief Ascription and the Ramsey Test'' at the 9th National Conference of the Italian Society for Analytic Philosophy (\href{http://www.filosofia.lettere.unipd.it/analitica/sifa2010/}{SIFA}) ``Truth, Knowledge and Science'', Padua, Italy, September 2010. 
    % 25 September 2010. 
    %23-25 September, http://www.filosofia.lettere.unipd.it/analitica/sifa2010/
  
    \ind ``Belief Ascription and the Ramsey Test'' at the International Congress of Italian Society for Logic and Philosophy of Science (\href{http://dinamico2.unibg.it/silfs/convegno2010.htm}{SILFS}), Bergamo, Italy, December 2010. 
    % 16 December 2010. 
    %15-17 December, http://dinamico2.unibg.it/silfs/convegno2010.htm

	\ind Poster: ``What Is Going On In the Nearset Possible World\ldots,'' Meeting of the European Society for Philosophy and  Psychology (\href{http://www.ruhr-uni-bochum.de/philosophy/espp2010/index.html}{ESPP2010}), Bochum and Essen, August 2010.

	\ind Comment on Benjamin Hoffmann's paper: ``Belief Revision for Dynamic Reliability Ranks,'' \href{http://formalphilosophy.org/node/580}{The Bi-Annual Konstanz--Leuven Workshop in Formal Epistemology}, Leuven, January 2010.
  	% 28 January 2010.

  
  

<<<<<<< HEAD
\bigskip 
=======
  \headtwo{Other presentations:}
  \begin{itemize}
  \item Comment on  ``A Causal Understanding of When and When Not to Jeffrey Conditionalize'' by Ben Schwan and Reuben Stern. \href{http://www.philos.rug.nl/few2016/}{Formal Epistemology Workshop 2016}, Groningen, 20 June 2016.

>>>>>>> origin/master

\marginhead{\sffamily {\vskip 0.2em}seminars and colloquia}
\medskip

    \ind Invited talk: ``What wrong with the Ramsey Test and why we should care,'' at the \href{http://www.mcmp.philosophie.uni-muenchen.de/events/weekly_talks_new/index.html}{MCMP Colloquium in Philosophy, Logic and Philosophy of Science}, LMU Munich, Germany, May 2015.
    % 13 May 2015.

    \ind Invited talk: ``Between \emph{if} and \emph{then}. On what is wrong with the Ramsey Test and how to fix it,'' Reasoning and Argumentation Lab (\href{http://www.bbk.ac.uk/psychology/ral}{ReAL}) Seminar, Birkbeck College London, UK, March 2015.
    % 9 March 2015

	\ind ``Indicatives and a~Problem of a~Bad Advice,'' WiP Seminar, Faculty of Philosophy, University of Groningen, March 2013.
	% 25 March 2013.

    \ind Invited talk: ``Conditionals and inferences'' at Logic and Interactive Rationality (\href{http://www.illc.uva.nl/lgc/seminar/}{LIRa}) Seminar at the Institute of Logic, Language, and Computation, University of Amsterdam, The Netherlands, March 2013.
    % 7 March 2013.

    \ind ``Truth Conditions for Conditional Sentences, Allan Gibbard and Ambiguity,'' PCCP Seminar, Faculty of
    Philosphy, University of Gronigen, September 2011.
    % 22 September 2011.
    
  


\bigskip 

<<<<<<< HEAD
\marginhead{\sffamily {\vskip 0.2em}teaching}
\medskip
=======
  % \headtwo{}
  % \begin{itemize}
  % \item Socrates-Erasmus stipend to
  %   \href{http://www.kuleuven.be}{Katholieke Universiteit Leuven},
  %   Belgium (ten month stipend for the academic year 2008 --- 2009).
  % \item Scholarship for academic performance in the academic year
  %   2007 --- 2008.
  % \end{itemize}
  
  
  \headtwo{Teaching:}
  \begin{itemize}
    \item Courses:
      \begin{itemize}
      	\item Advanced seminar \emph{Philosophy of Language} (Summer 2016/2017, LMU Munich)
        \item Advanced seminar \emph{History of the Analytic Philosophy} (Summer 2015/2016, LMU Munich).
        \item Advanced seminar \emph{Philosophy of Mind} (Winter 2015/2016, LMU Munich).
        \item Advanced seminar \emph{Conditionals and Psychology of Reasoning} (Summer 2014/2015, LMU Munich).
        \item Master's level seminar \emph{Philosophy of Logic: Conditionals}, together with I.~Douven (Summer 2013, University of Groningen).
     \end{itemize}
    \item Lectures:
      \begin{itemize}
         \item MCMP Fellows' session \emph{Learning That If P Then Q} (Third Summer School on Mathematical Philosophy for Female Students, July 2016, LMU Munich)
         \item PostDoc session \emph{Conditionals and Cognitive Science} (Second Summer School on Mathematical Philosophy for Female Students, July 2015, LMU Munich)
      \end{itemize}
  \end{itemize}
  
>>>>>>> origin/master

\noindent LMU Munich:

<<<<<<< HEAD
\medskip

\ind \emph{Philosophy of Language}, advanced seminar, Summer 2017.

\ind \emph{History of the Analytic Philosophy}, advanced seminar, Summer 2016.

\ind \emph{Philosophy of Mind}, advanced seminar, Winter 2015/2016.

\ind \emph{Conditionals and Psychology of Reasoning}, advanced seminar, Summer 2015.

\medskip

\noindent University of Groningen:

\medskip

\ind \emph{Philosophy of Logic: Conditionals}, master's level seminar, Summer 2013 (together with Igor Douven).

\medskip

\noindent Summer Schools:

\medskip

\ind \emph{Learning That If P Then Q}, a lecture at the MCMP Fellows' session of the Third Summer School on Mathematical Philosophy for Female Students, July 2016, LMU Munich.

\ind \emph{Conditionals and Cognitive Science} a lecture at the PostDoc's session of the Second Summer School on Mathematical Philosophy for Female Students, July 2015, LMU Munich.


\bigskip

\marginhead{\sffamily {\vskip 0.15em}service to the profession}
% \marginhead{\sffamily {\vskip 0.15em}service \newline  to the profession}
\medskip

\ind \emph{Co-organised events:}
	
	\medskip
	\ind \href{http://www.cas.uni-muenchen.de/veranstaltungen/tagungen/ws_krzyzanowska_hartmann/index.html}{Learning Conditionals Workshop} (with Stephan Hartmann; CAS, LMU Munich, 2-3 February 2017.)
 
	\ind \href{http://www.mathsummer.philosophie.uni-muenchen.de}{Third Summer School on Mathematical Philosophy for Female Students} (with Samuel Fletcher and Milena Ivanova; MCMP, LMU Munich, 24-30 July 2016).

	\ind Contributed Symposium \emph{If, Then, Otherwise: A Symposium on Conditionals}, the 23rd Annual Meeting of the European Society for Philosophy and Psychology (\href{http://espp2015.ut.ee}{ESPP2015}), Tartu, Estonia, 14-17 July 2015.

	\ind \href{http://lmu.de/cpr2015}{Causal and Probabilistic Reasoning Workshop} (with Stephan Hartmann, Gregory Wheeler and Michael Waldmann; MCMP, LMU Munich, 18-20 June 2015)

	\ind \href{http://www.philos.rug.nl/GCTP2013/}{Graduate Conference in Theoretical Philosophy} (with PhD students in Theoretical Philosophy; University of Groningen, 18-20 April 2013).

		\bigskip
\ind \emph{Refereeing:}
		
	\medskip
		\ind \emph{\emph{Analysis}, \emph{Cognition}, \emph{Journal of Cognitive Psychology}, \emph{Journal of Cognitive Science}, \emph{Episteme}, \emph{Minds \& Machines}, \emph{CogSci2015}, \emph{CogSci2016}, \href{https://www.ncn.gov.pl/}{National Science Center} (Polish Research Funding Agency)}.

		\bigskip
\ind \emph{Program Committee Memberships:}
		
		\medskip
		\ind \emph{Formal Epistemology Workshop 2017} (University of Washington), \emph{Formal Epistemology Workshop 2016} (RUG), \emph{Semantics of Theories Conference} (MCMP, LMU Munich), \emph{2nd Munich Graduate Workshop in Mathematical Philosophy: Formal Epistemology} (MCMP, LMU Munich).

    \bigskip
\ind \emph{Service to the University:}

		\medskip
		\ind Selection committees for PhD/postdoctoral fellowships and visiting fellowships at the Munic Center for Mathematical Philosophy, LMU Munich.

		\ind Second reader of the M.A. Thesis by Yunhao Zhang ``On considering Psychology Scientific: How a Discipline with 54 Subfields Operates as Science.''

\bigskip

\marginhead{\sffamily {\vskip 0.2em}other}
\medskip

\ind Languages: Polish (mother tongue), English (full professional proficiency), Dutch
    (basic speaking, intermediate reading), German (intermediate speaking, advanced reading, Zentrale
    Mittelstufe Pr\"ufung 2006). 

\ind Technical skills: \LaTeX$_{2e}$, basics of (X)HTML/CSS, basics of the R environment for statistical
=======
  \headtwo{Activities:}
  \begin{itemize}
  	\item (Co)organization of events:
  		\begin{itemize}
		  \item \href{http://www.mathsummer.philosophie.uni-muenchen.de}{Third Summer School on Mathematical Philosophy for Female Students} (with Samuel Fletcher and Milena Ivanova; MCMP, LMU Munich, 24-30 July 2016).
		  \item Contributed Symposium \emph{If, Then, Otherwise: A Symposium on Conditionals}, the 23rd Annual Meeting of the European Society for Philosophy and Psychology (\href{http://espp2015.ut.ee}{ESPP2015}), Tartu, Estonia, 14-17 July 2015.
		  \item \href{http://lmu.de/cpr2015}{Causal and Probabilistic Reasoning Workshop} (with Stephan Hartmann, Gregory Wheeler and Michael Waldmann; MCMP, LMU Munich, 18-20 June 2015)
		  \item  \href{http://www.philos.rug.nl/GCTP2013/}{Graduate Conference in
	      Theoretical Philosophy} (with PhD students in Theoretical Philosophy; University of Groningen, 18-20 April 2013).
  		\end{itemize}
  	\item Research visits: Birkbeck College London, March 2015.
  		% \begin{itemize}
  		% 	\item 
  		% \end{itemize}

    \item Reviewing: \emph{Cognition}, \emph{Journal of Cognitive Science}, \emph{Minds \& Machines}, \emph{CogSci2015}, \emph{CogSci2016}.
    
    \item Program Committee Memberships: \emph{Formal Epistemology Workshop 2016} (RUG), \emph{Semantics of Theories Conference} (MCMP, LMU Munich), \emph{2nd Munich Graduate Workshop in Mathematical Philosophy: Formal Epistemology} (MCMP, LMU Munich).
  \end{itemize}



  \headtwo{Miscellaneous:}
  \begin{itemize}
  \item Languages: Polish (mother tongue), English (full professional proficiency), Dutch
    (basic speaking, advanced reading), German (basic speaking, advanced reading, Zentrale
    Mittelstufe Pr\"ufung 2006)
  \item Computer skills: \LaTeX$_{2e}$, basics of
    (X)HTML/CSS, basics of the R environment for statistical
>>>>>>> origin/master
    computing.

\ind Hobbies: playing cello, sport climbing, painting.

\bigskip

\marginhead{\sffamily {\vskip 0.15em}references}
% \marginhead{\sffamily {\vskip 0.15em}service \newline  to the profession}
\medskip

\ind Prof. Igor Douven\\
Directeur de Recherche at the CNRS\\ 
Laboratoire Sciences, Normes et Décisions, Université Paris-Sorbonne\\
% Maison de la Recherche, 28 rue Serpente, 75006 Paris, France\\
\href{mailto:igor.douven@paris-sorbonne.fr}{igor.douven@paris-sorbonne.fr}\\

\ind Prof. David Over\\
Department of Psychology, Durham University\\ 
% Durham, United Kingdom.
\href{mailto:david.over@durham.ac.uk}{david.over@durham.ac.uk}\\

\ind Prof. Hans Rott\\
Lehrstuhl für Theoretische Philosophie\\
Institut für Philosophie, Universität Regensburg\\
\href{mailto:hans.rott@psk.uni-regensburg.de}{hans.rott@psk.uni-regensburg.de}

  % \headtwo{References:}
  % \begin{itemize}
  % \item Prof.\ Igor Douven\\
  % \href{mailto:igor.douven@paris-sorbonne.fr}{\texttt{igor.douven@paris-sorbonne.fr}}\\
  % French National Centre for Scientific Research\\
  % Institut des Sciences Humaines et Sociales (INSHS)\\
  % Sciences, Normes, Décision, Université Paris-Sorbonne\\
  % Paris, France.

  % \item Prof.\ David E. Over\\
  % \href{mailto:david.over@durham.ac.uk}{\texttt{david.over@durham.ac.uk}}\\
  % Durham University, Department of Psychology\\
  % Durham, United Kingdom.

  % \end{itemize}



\end{document}

%To add one day:
Research visits:







