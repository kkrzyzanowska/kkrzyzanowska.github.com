\documentclass[a4paper,12pt]{article}

\usepackage{ucs}
\usepackage[T1]{polski}
\usepackage[utf8x]{inputenc}
\usepackage{geometry}
\usepackage{graphicx}
%\usepackage{beramono}

\pagestyle{empty}

\newenvironment{sekcja}
{\begin{tabular}{r|l}}
{\end{tabular}}

%\usepackage{hyperref}
\usepackage{color}
\usepackage[usenames,dvipsnames]{xcolor}
\usepackage[final, %override "draft which means "do nothing"
colorlinks, %rather than outlining them in boxes
linkcolor=Sepia, %override truly awful colour choices
citecolor=Sepia, %(ditto)
%urlcolor=Bittersweet,
urlcolor=RedOrange,
plainpages=false %(ditto)
]{hyperref}

\hypersetup{%
  pdftitle={Curriculum Vitae},
  pdfauthor={Karolina Krzyzanowska},
 % colorlinks={true},
 % linkcolor={pink}
}

\setlength{\parindent}{0pt}
\geometry{a4paper, lmargin=2cm, rmargin=2cm, bmargin=2cm, tmargin=2cm}
\begin{document}

\newcommand{\headone}[1] {
 \section*{#1}
\rule[2.5ex]{\textwidth}{2pt}}

\newcommand{\headtwo}[1]{
  \subsection*{#1}
  \rule[1.5ex]{\textwidth}{.5pt}}

\newcommand{\headthree}[1]{ {\large #1}}

\newcommand{\link}[1]{ \texttt{\href{#1}{#1}}}

\begin{small}

  \headone{Curriculum Vit\ae}
  
  Karolina Krzyżanowska\\
  \href{mailto:k.krzyzanowska@lmu.de}{\texttt{k.krzyzanowska@lmu.de}}\\
  Munich Center for Mathematical Philosophy\\
  Ludwig-Maximilians-Universität München\\
  Fakultät für Philosophie, Wissenschaftstheorie und Religionswissenschaft\\
  Geschwister-Scholl-Platz 1, 80539 München, Germany.\\
  % Visiting address: Ludwigstraße 31, Room 131\\
  Homepage: \href{http://karolinakrzyzanowska.com}{\texttt{http://karolinakrzyzanowska.com}}\\


  
  \headtwo{Academic interests:}
  \begin{itemize}
  \item AOS: Conditionals, philosophy of language, psychology of reasoning.
  \item AOC: Epistemology, philosophical logic, cognitive science,
    philosophy of psychology.
  \end{itemize}
  
  \headtwo{Academic work experience:}
  \begin{itemize}
    \item Postdoctoral research fellow at the \emph{\href{http://www.mcmp.philosophie.uni-muenchen.de/}{Munich Center for Mathematical Philosophy}}\\
    \emph{\href{http://www.en.uni-muenchen.de/}{Ludwig-Maximilian-Universität München}, Germany}\\
    October 2014 --- Present.

    \item Visiting researcher 
    % in the \href{http://www.uib.no/en/rg/lii}{Logic, Information and Interaction} group 
    at the Department of Information Science and Media Studies\\ 
    \emph{\href{http://www.uib.no/en}{University of Bergen}, Norway}\\
    June 2014 --- September 2014.

    \item Researcher in the \href{http://www.pfeifer-research.de/spp.html}{Project PF 740/2-1} ``Rational
    reasoning with conditionals and probabilities. Logical foundations and empirical evaluations''
    within the DFG Priority Program SPP 1516 ``New Frameworks of Rationality'' at the \emph{\href{http://www.mcmp.philosophie.uni-muenchen.de/}{Munich Center for Mathematical Philosophy}}\\
    \emph{\href{http://www.en.uni-muenchen.de/}{Ludwig-Maximilian-Universität München}, Germany}\\
    Principal Investigator: Dr. Dr. Niki Pfeifer\\
    February 2014 --- July 2014.

    \item PhD student at
    the %\href{http://formalphilosophy.org/}{Formal Epistemology Project}\\
    \emph{Department of Theoretical Philosophy, \href{http://www.rug.nl/filosofie/}{Faculty of Philosophy}\\
    \href{http://www.rug.nl}{University of Groningen}, The Netherlands}\\
    % \href{http://formalphilosophy.org/}{Formal Epistemology
    % Project}\\	
    Supervisor: Prof. Igor Douven\\
    November 2010 --- October 2013.


    \item PhD student at the  \href{http://formalphilosophy.org/}{Formal Epistemology Project}\\
    \emph{Centre for Logic and Analytic Philosophy, \href{http://hiw.kuleuven.be/eng/}{Institute of Philosophy}\\
    \href{http://www.kuleuven.be/english}{Katholieke Universiteit Leuven}, Belgium}\\
    Supervisor: Prof. Igor Douven\\
    November 2009 --- October 2010.

  \end{itemize}


  
  \headtwo{Education:}
  \begin{itemize}
    \item PhD student at
    the %\href{http://formalphilosophy.org/}{Formal Epistemology Project}\\
    \emph{Department of Theoretical Philosophy, \href{http://www.rug.nl/filosofie/}{Faculty of Philosophy}\\
    \href{http://www.rug.nl}{University of Groningen}, The Netherlands}\\
    % \href{http://formalphilosophy.org/}{Formal Epistemology
    % Project}\\
    Degree: Ph.D.\ in Philosophy\\
    Thesis: \emph{\href{http://karolinakrzyzanowska.com/pdfs/krzyzanowska-phd-final.pdf}{Between ``if'' and ``then.'' Towards an empirically informed philosophy of conditionals.}} 
    November 2010 --- February 2015.\\
    Supervisor: Prof.\ Igor Douven  
    
    \item \emph{\href{http://www.filozofia.uw.edu.pl}{Institute of Philosophy}, \href{http://www.uw.edu.pl}{University of Warsaw}, Poland}\\
    October 2004 --- September 2009\\
    Degree: M.A.\ in Philosophy\\
    Thesis: \emph{Belief Reports in Linguistic and Cognitive Perspective} (in Polish)\\
    Advisor: Dr.\ Justyna Grudzińska.
    
    \item \emph{\href{http://hiw.kuleuven.be/eng/}{Institute of Philosophy}, \href{http://www.kuleuven.be}{Katholieke Universiteit Leuven}, Belgium}\\
    Erasmus Exchange Programme\\
    September 2008 --- June 2009.
  

    \item Summer schools and workshops:
    \href{http://www.esslli2012.pl}{ESSLLI 2012} in Opole, Poland;
      %%
    \href{http://esslli2011.ijs.si}{ESSLLI 2011} in Ljubljana, Slovenia;
      %%
    % Workshop on
    % \href{http://www.phil-fak.uni-duesseldorf.de/conditionals/}{Conditionals,
    % Counterfactuals and Causes in Uncertain
    % Environments} 2011 in Düsseldorf, Germany;
      %%
    ESSLLI 2010 in Copenhagen, Denmark;
      %%
    % \href{http://www.bbk.ac.uk/psychology/our-research/londonreasoningworkshop/lrw5}{5th
    % LRW} 2010 at Birkbeck College London, UK;
    %   %%
    % \href{http://homepages.ulb.ac.be/~uicm3/}{UICM III} 2010 in
    % Brussels, Belgium;
      %%
    Logic Workshop \href{http://www.logika.uw.edu.pl/warsztaty2007/index.html}{``Logic
      and Cognition''} 2007 on Szrenica, Szklarska Poręba, Poland.

    % \item \emph{High School: Liceum Ogólnokształcące im. Bolesława Prusa, Skierniewice, Poland}\\
    % September 2000 --- June 2004.

  \end{itemize}


  \headtwo{Published papers:}
  \begin{itemize}
    \item `Rethinking Gibbard's Riverboat Argument' with Sylvia Wenmackers and Igor Douven, \emph{Studia Logica}, Vol. 102, No. 4, pp. 771-792, 2014.
    % \href{http://link.springer.com/article/10.1007/s11225-013-9507-2}

    \item `Inferential Conditionals and Evidentiality' with Sylvia Wenmackers and Igor Douven, \emph{Journal of Logic, Language and Information}, Vol.~22 No.~3, pp.~315–334, 2013.
    % \href{http://link.springer.com/article/10.1007/s10849-013-9178-4}{}

    \item `Belief Ascription and the Ramsey Test', \emph{Synthese} Vol.~190, No.~1, pp.~21-36, 2013. % DOI:
    % \href{http://www.springerlink.com/content/6r04w05l4l457485/}{10.1007/s11229-012-0160-5}.

    \item `Conditionals, Inference, and Evidentiality' with Sylvia Wenmackers, Igor Douven and Sara Verbrugge, in: Jakub Szymanik and Rineke Verbrugge (eds.): \href{http://ceur-ws.org/Vol-883/}{`Proceedings of the Logic \& Cognition Workshop at ESSLLI 2012, Opole, Poland, 13-17 August, 2012'}, vol.~883 of CEUR Workshop Proceedings, CEUR-WS.org, pp.~38-47, 2012.

    \item `Ambiguous Conditionals' in: Piotr Stalmaszczyk (ed.): `Philosophical and Formal Approaches to Linguistic Analysis', Ontos Verlag, pp.~315-332, 2012.

    \item `Sprawozdania z przekonań w perspektywie filozofii języka i kognitywistyki' (`Belief Reports in the perspective of philosphy of language and cognitive science'), \emph{Przegląd Filozoficzny} No.~3 (75), pp.~297-319, 2010.

  \end{itemize}

  \headtwo{Forthcoming papers:}
  \begin{itemize}
   \item `Deliberationally useless conditionals,' manuscript under revision.
    
    \begin{description}{\footnotesize
      \item[Abstract:] \textit{Decision theorists tend to treat indicative conditionals with reservation, as they can be sources of bad or even conflicting advice. Contrary to the mainstream view, Keith DeRose argued that indicatives can be conditionals of deliberation, though some of them are unassertable in the contexts of de- liberation. However, as we are going to argue, DeRose failed to recognise the reason why this is the case. We will show that a recently proposed ``inferential'' account of indicative conditionals provides a more straightforward explanation of this phenomenon and, simultaneously, a way of telling ``deliberationally useless'' and ``deliberationally useful'' conditionals apart.}}
    \end{description}

   
   \item `Antecedents, consequents and something in between,' in preparation.
    \begin{description}{\footnotesize
      \item[Abstract:] \textit{It is a common intuition that the antecedent of an indicative conditional should have something to do with its consequent, that they should be somehow connected. However, only very few semantic theories of conditionals do justice to this intuition and the majority tends to relegate it to the realm of pragmatics. Yet no one has offered a full-fledged pragmatic account of the oddness of missing-link conditionals. The aim of this paper is to discuss possible pragmatic explanations of the phenomenon and, consequently, to show how they fail.}}
    \end{description}

  \item `Approximate interpretation of numerals: number sense and semantics/pragmatics interface,' in preparation (with Paula Quinon).
    \begin{description}{\footnotesize
      \item[Abstract:] \textit{It is a well known fact that expressions like ``5 million,'' ``two hundreds'' or even, in some contexts, numerals denoting smaller numbers like ``forty'' tend to be interpreted as approximations. Although the pragmatic aspects of this phenomenon has been widely discussed in the linguistic literature, the semantics of sentences containing round numbers has not attracted much attention. Moreover, it is far from obvious what is the truth value of a sentence like ``Norway has a population of 5 million,'' even if we know that the exact number of the country's inhabitants is 5,109,059. Can a sentence be assertable yet false? Drawing from recent developments in cognitive science, we argue that our preference for vague interpretation of numerals is due to the approximate number system being the primary source of our mental representation of numbers. Furthermore, we offer a conceptual spaces model of the interpretation of round numbers.}}
    \end{description}



   \item `Persuading with conditionals,' in preparation (with Ulrike Hahn and Peter Collins).
   % \item Empirical evaluation of different solutions to the old evidence problem (with Stephan Hartmann).
   \item An experimental study on implicatures and their effect on people's assertability, believability, and truth value judgments (with Igor Douven and Henrik Singmann).
   \item `What psychology of reasoning can and cannot tell us about the semantics of conditionals?' in preparation.  
  \end{itemize}
  

  \headtwo{Invited talks:}
  \begin{itemize} 

    \item \emph{What psychology of reasoning can teach us about the meaning of indicative conditionals: The case of a metalinguistic theory} at the \href{http://www.f.bg.ac.rs/fil-konf/konferencija2-home.html}{2nd Belgrade Conference on Conditionals}, Belgrade, Serbia, 28-31  May 2016.

    \item Keynote talk: \emph{What Psychology of Reasoning Can Tell Us About the Meaning of Indicative Conditionals} at the \href{http://logicalconnectives.uw.edu.pl/}{Warsaw Workshop in Philosophy of Language: Logic and Meaning}, Warsaw, Poland, 13-14 May 2016.

    \item \emph{What are indicative conditionals about?} at EuNoC Workshop, Leeds, UK, 19-20 January 2016.

    \item \emph{A new kind of metalinguistic theory of conditionals} at the workshop \href{http://cms.uni-konstanz.de/what-if/events/workshop-in-osnabrueck-september-2015-p1-and-p7/}{``Do Conditionals have Truth Conditions? Dynamic and Pragmatic Aspects of Conditionals''}, GAP.9 congress, Osnabrück, Germany, 18-19 September 2015.

    \item \emph{What wrong with the Ramsey Test and why we should care} at the \href{http://www.mcmp.philosophie.uni-muenchen.de/events/weekly_talks_new/index.html}{MCMP Colloquium in Philosophy, Logic and Philosophy of Science}, LMU Munich, Germany, 13 May 2015.

    \item \emph{Between ``if'' and ``then.'' On what is wrong with the Ramsey Test and how to fix it} at the Reasoning and Argumentation Lab (\href{http://www.bbk.ac.uk/psychology/ral}{ReAL}) Seminar, Birkbeck College London, UK, 9 March 2015.

    \item \emph{Conditionals and inferences} at Logic and Interactive
    Rationality (\href{http://www.illc.uva.nl/lgc/seminar/}{LIRa})
    Seminar at the Institute of Logic, Language, and Computation,
    University of Amsterdam, The Netherlands, 7 March 2013.
  
    \item \emph{Inferential Conditionals} at
    \href{https://sites.google.com/site/jannekehuitink/workshop}{Conditionals Workshop}, Faculty of Philosophy, University of Groningen, The Netherlands, 9 November 2012.
  \end{itemize}
  
  \headtwo{Contributed talks:}
  \begin{itemize}
    \item \emph{Between ``if'' and ``then.'' What do we learn when we learn a conditional?} at the Symposium \emph{If, Then, Otherwise: A Symposium on Conditionals}. 23rd Annual Meeting of the European Society for Philosophy and Psychology (\href{http://espp2015.ut.ee}{ESPP2015}), Tartu, Estonia, 14-17 July 2015.

    \item \emph{The Ramsey Test and Conditionals of a Bad Advice} at the Annual Meeting of the Priority Program ``New Frameworks of Rationality'' (\href{http://www.spp1516.de}{SPP1516}), Etelsen, Germany, 15-18 March 2015.

    \item \emph{Exact numerals as vague quantifiers} at the \href{https://sites.google.com/site/szklarskaporebaworkshop16/}{16th Szklarska Poręba Workshop} on the Roots of Pragmasemantics, Szrenica, Szklarska Poręba, Poland, 20-23 February 2015.

    \item \emph{The Variety of Conditionals and the Evidential Function of Epistemic Modals} at The 7th London Reasoning Workshop (\href{http://www.bbk.ac.uk/psychology/about-us/events/the-7th-london-reasoning-workshop}{LRW}), Birkbeck College London, UK, 25-26 July 2013.

    \item \emph{Deliberationally Useless Conditionals} at the 21st Annual Meeting of the European Society for Philosophy and Psychology (\href{http://espp2013.com}{ESPP2013}), Granada, Spain, 9-12 July 2013.

    \item \emph{Rethinking Gibbard’s Riverboat Argument} at the 8th Barcelona Workshop on Conditionals
    (\href{http://www.ub.edu/logosbw/bw8/index.html}{BW8}), Universitat de Barcelona, Spain, 26-28 June 2013.
    
    \item \emph{What ``Must'' and ``Should'' Can Mean} at the Graduate Conference in Theoretical Philosophy
    (\href{http://www.philos.rug.nl/GCTP2013/}{GCTP2013}), Groningen, The Netherlands, 19 April 2013. 
    %18-20 April, http://www.philos.rug.nl/GCTP2013/
    
    \item \emph{Conditionals, Inference, and Evidentiality} at the \href{http://www.ai.rug.nl/SocialCognition/logic-cognition/}{Logic \& Cognition Workshop}, European Summer School for Logic, Language, and Information (\href{http://www.esslli2012.pl}{ESSLLI}), Opole, Poland, 15 August 2012. 
    %13-17 August, http://www.ai.rug.nl/SocialCognition/logic-cognition/

    \item \emph{Ambiguity and Gibbard's Argument Against Propositional Theories of Conditionals} at the Second International Conference on Philosophy of Language and Linguistics (\href{http://ia.uni.lodz.pl/linguistics/events/philang-2011}{PhiLang2011}),  Department of English and General Linguistics, Łódź, Poland
    13 May 2011. 
    %12-14 May, http://ia.uni.lodz.pl/linguistics/events/philang-2011

    \item \emph{Belief Ascription and the Ramsey Test} at the 9th National Conference of the Italian Society for Analytic Philosophy (\href{http://www.filosofia.lettere.unipd.it/analitica/sifa2010/}{SIFA}) ``Truth, Knowledge and Science'', Padua, Italy, 25 September 2010. 
    %23-25 September, http://www.filosofia.lettere.unipd.it/analitica/sifa2010/
  
    \item \emph{Belief Ascription and the Ramsey Test} at the International Congress of Italian Society for Logic and Philosophy of Science (\href{http://dinamico2.unibg.it/silfs/convegno2010.htm}{SILFS}), Bergamo, Italy, 16 December 2010. 
    %15-17 December, http://dinamico2.unibg.it/silfs/convegno2010.htm
  \end{itemize}
  

  \headtwo{Other presentations:}
  \begin{itemize}

  \item Comment on Matteo Colombo's presentation \emph{Explanatory Reasoning, Moral Value, and Economic Incentive. Or\ldots How much is your moral judgment?} 
    \href{http://www.psychologie.uni-freiburg.de/Members/singmann/operational2013}{Operationalization: An Interdisciplinary Workshop at the Edge of Experimental Psychology and Analytical Philosophy}, Freiburg, 16 October 2013.

  \item \emph{Inferential Conditionals and the Meaning of Some
      Epistemic Modals} (poster). Sixth Semantics and Philosophy
    in Europe Colloquium
    (\href{http://spe6conference.wordpress.com}{SPE6}),
    Saint-Petersburg State University, 10-14 June 2013.
    
  \item \emph{Indicatives and a~Problem of a~Bad Advice}. WiP
    Seminar, Faculty of Philosophy, University of Groningen, 25 March
    2013.
    
  \item \emph{What Inferential Conditionals Can Reveal About Epistemic
      Modals: An Experimental Study Of Evidentiality} (poster).
    3rd Workshop of Experimental Philosophy Group UK:
    \href{https://www.nottingham.ac.uk/philosophy/research/conferences/workshop-intuitions-experimentsandphilosophy.aspx}{Inuitions, Experiments, and Philosophy}, Nottingham, September 2012.

  \item \emph{Truth Conditions for Conditional Sentences, Allan
      Gibbard and Ambiguity}. PCCP Seminar, Faculty of
    Philosphy, University of Gronigen, 22 September 2011.
    
  \item \emph{What Is Going On In the Nearset Possible World\ldots} (poster). 
  Meeting of the European Society for Philosophy and  Psychology
    (\href{http://www.ruhr-uni-bochum.de/philosophy/espp2010/index.html}{ESPP2010}),
    Bochum and Essen, August 2010.

  \item Comment on Benjamin Hoffmann's paper \emph{Belief Revision for
      Dynamic Reliability Ranks}. The Bi-Annual Konstanz--Leuven
    \href{http://formalphilosophy.org/node/580}{Workshop in Formal
      Epistemology}, Leuven, 28 January 2010.
  \end{itemize}

  



  % \headtwo{}
  % \begin{itemize}
  % \item Socrates-Erasmus stipend to
  %   \href{http://www.kuleuven.be}{Katholieke Universiteit Leuven},
  %   Belgium (ten month stipend for the academic year 2008 --- 2009).
  % \item Scholarship for academic performance in the academic year
  %   2007 --- 2008.
  % \end{itemize}
  
  
  \headtwo{Teaching:}
  \begin{itemize}
    \item Courses:
      \begin{itemize}
        \item Advanced seminar \emph{History of the Analytic Philosophy} (Summer 2015/2016, LMU Munich).
        \item Advanced seminar \emph{Philosophy of Mind} (Winter 2015/2016, LMU Munich).
        \item Advanced seminar \emph{Conditionals and Psychology of Reasoning} (Summer 2014/2015, LMU Munich).
        \item Master's level seminar \emph{Philosophy of Logic: Conditionals}, together with I.~Douven (Summer 2013, University of Groningen).
     \end{itemize}
    \item Lectures:
      \begin{itemize}
         \item MCMP Fellows' session \emph{Learning That If P Then Q} (Third Summer School on Mathematical Philosophy for Female Students, July 2016, LMU Munich)
         \item PostDoc session \emph{Conditionals and Cognitive Science} (Second Summer School on Mathematical Philosophy for Female Students, July 2015, LMU Munich)
      \end{itemize}
  \end{itemize}
  


  \headtwo{Activities:}
  \begin{itemize}
  	\item (Co)organization of events:
  		\begin{itemize}
		  \item \href{http://www.mathsummer.philosophie.uni-muenchen.de}{Third Summer School on Mathematical Philosophy for Female Students} (with Samuel Fletcher and Milena Ivanova; MCMP, LMU Munich, 24-30 July 2016).
		  \item Contributed Symposium \emph{If, Then, Otherwise: A Symposium on Conditionals}, the 23rd Annual Meeting of the European Society for Philosophy and Psychology (\href{http://espp2015.ut.ee}{ESPP2015}), Tartu, Estonia, 14-17 July 2015.
		  \item \href{http://lmu.de/cpr2015}{Causal and Probabilistic Reasoning Workshop} (with Stephan Hartmann, Gregory Wheeler and Michael Waldmann; MCMP, LMU Munich, 18-20 June 2015)
		  \item  \href{http://www.philos.rug.nl/GCTP2013/}{Graduate Conference in
	      Theoretical Philosophy} (with PhD students in Theoretical Philosophy; University of Groningen, 18-20 April 2013).
  		\end{itemize}
  	\item Research visits: Birkbeck College London, March 2015.
  		% \begin{itemize}
  		% 	\item 
  		% \end{itemize}

    \item Reviewing: \emph{Cognition}, \emph{Journal of Cognitive Science}, \emph{Minds \& Machines}, \emph{CogSci2015}, \emph{CogSci2016}.
    
    \item Program Committee Memberships: \emph{Formal Epistemology Workshop 2016} (RUG), \emph{Semantics of Theories Conference} (MCMP, LMU Munich), \emph{2nd Munich Graduate Workshop in Mathematical Philosophy: Formal Epistemology} (MCMP, LMU Munich).
  \end{itemize}



  \headtwo{Miscellaneous:}
  \begin{itemize}
  \item Languages: Polish (mother tongue), English (full professional proficiency), Dutch
    (basic speaking, advanced reading), German (basic speaking, advanced reading, Zentrale
    Mittelstufe Pr\"ufung 2006)
  \item Computer skills: \LaTeX$_{2e}$, basics of
    (X)HTML/CSS, basics of the R environment for statistical
    computing.

  \item Hobbies: playing cello, sport climbing, painting.
  \end{itemize}

  % \headtwo{References:}
  % \begin{itemize}
  % \item Prof.\ Igor Douven\\
  % \href{mailto:igor.douven@paris-sorbonne.fr}{\texttt{igor.douven@paris-sorbonne.fr}}\\
  % French National Centre for Scientific Research\\
  % Institut des Sciences Humaines et Sociales (INSHS)\\
  % Sciences, Normes, Décision, Université Paris-Sorbonne\\
  % Paris, France.

  % \item Prof.\ David E. Over\\
  % \href{mailto:david.over@durham.ac.uk}{\texttt{david.over@durham.ac.uk}}\\
  % Durham University, Department of Psychology\\
  % Durham, United Kingdom.

  % \end{itemize}



\end{small}
\end{document}
